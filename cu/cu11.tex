\begin{UseCase}{CU11}{Activar Estado de Medicamento}{
		Por razones laborales es necesario activar el medicamneto para que se pueda permitir su venta en las sucursales.
	}
		\UCitem{Versión}{\color{Gray}0.1}
		\UCitem{Autor}{\color{Gray}Vázquez Cruz Fernando Darwin }
		\UCitem{Supervisa}{\color{Gray}}
		\UCitem{Actor}{\hyperlink{Alumno}{Dueño}}
		\UCitem{Propósito}{Permitir que un medicamento pueda ser vendido y distribuido en las sucursales.}
		\UCitem{Entradas}{Nombre del medicamento, código de barras del medicamento o ingrediente activo del medicamento.}
		\UCitem{Origen}{Teclado, Mouse}
		\UCitem{Salidas}{Mensaje de confirmación de activación.}
		\UCitem{Destino}{Pantalla}
		\UCitem{Precondiciones}{El medicamento debe de existir.}
		\UCitem{Postcondiciones}{Se activará el medicamento.}
		\UCitem{Errores}{El medicamento no puede ser activado.}
		\UCitem{Tipo}{Caso de uso primario.}
		\UCitem{Observaciones}{}
		\UCitem{Estado}{Revisión}
	\end{UseCase}
%--------------------------------------
	\begin{UCtrayectoria}{Principal}
		\UCpaso Se extiende del caso de uso \UCref{CU12} paso 6.
		\UCpaso[\UCactor] Da clic en la barra de búsqueda.
		\UCpaso[\UCactor] Selecciona el tipo de búsqueda que hará.
		\UCpaso [\UCactor] Ingresa nombre del medicamento, código de barras o ingrediente activo según corresponda. 
		\UCpaso[\UCactor] Da clic en el \Ibutton {+ buscar } .
		\UCpaso Despliega una lista que coincida con la búsqueda realizada. \Trayref{A}
		\UCpaso[\UCactor] Selecciona el medicamento de la lista.
		\UCpaso[\UCactor] Da clic en el \Ibutton {+ estado }.
		\UCpaso Despliega el mensaje \bf {+ MSG11 }.
		\UCpaso[\UCactor] El Dueño da clic en el \Ibutton {+ aceptar } para confirmar la operación. \Trayref{B}
		\UCpaso Actualiza el estado del medicamento y la lista de medicamentos.
		\UCpaso Re-direcciona al \UCactor a la \IUref{IU34} {Pantalla Principal de dueño}.
	
	\end{UCtrayectoria}


		\begin{UCtrayectoriaA}{A}{El Medicamento no se encuentra en la base de datos}
			\UCpaso[\UCactor] El Dueño busca el medicamento a seleccionar y no lo encuentra.
			\UCpaso[\UCactor] El Dueño regresa a la pantalla principal de medicamentos.
			\UCpaso Continua en el paso 4 del \UCref{CU11}.
		\end{UCtrayectoriaA}
		
		
		\begin{UCtrayectoriaA}{B}{El Medicamento ya está activado}.
			\UCpaso[\UCactor] El Dueño da clic en el \Ibutton {+ estado }.
			\UCpaso Despliega el mensaje \bf {+ MSG11 }.
			\UCpaso[\UCactor] El Dueño lee el mensaje y se percata que el medicamento ya está activado.
			\UCpaso[\UCactor] El Dueño da clic en el \Ibutton {+ cancelar }.
			\UCpaso Continua en el paso 4 del \UCref{CU11}.
		\end{UCtrayectoriaA}

%--------------------------------------