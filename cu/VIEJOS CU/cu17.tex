\begin{UseCase}{CU17}{Cambiar Datos de Clientes}
{
    Se requiere cambiar los datos de clientes de forma ocasional. Ya sea porque hay un error en los
    datos registrados de este o porque el cliente ha perdido su tarjeta y tiene que ser reasignado a
    una nueva.
}
    \UCitem{Versión}
    {
	\color{Gray} 0.1
    }
    \UCitem{Autor}
    {
	\color{Gray} Miguel \'Angel Mart\'inez Quevedo
    }
    \UCitem{Supervisa}
    {
	\color{Gray}
    }
    \UCitem{Actor}
    {
	Empleado
    }
    \UCitem{Propósito}
    {
	Corregir errores en los registros actuales de los clientes.
	Realizar cambios de tarjeta cambiando el n\'umero de tarjeta registrado.
    }
    \UCitem{Entradas}
    {
	Cliente a Cambiar. Datos a cambiar. Cambios.
    }
    \UCitem{Origen}
    {
	Teclado, Mouse
    }
    \UCitem{Salidas}
    {}
    \UCitem{Destino}
    {}
    \UCitem{Precondiciones}
    {
	El usuario eligi\'o la opci\'on de Cambiar Datos de Clientes del men\'u principal.
    }
    \UCitem{Postcondiciones}
    {
	El registro del usuario est\'a cambiado de acuerdo a lo que requer\'ia el usuario.
    }
    \UCitem{Errores}
    {}
    \UCitem{Historia de cambio}
    {}
    \UCitem{Observaciones}
    {}
    \UCitem{Estado}
    {
		Aprobado
	}
\end{UseCase}

%Trayectoria Principal
\begin{UCtrayectoria}{Principal}
    \UCpaso Se extiende del caso de uso \UCref{CU0} paso 11.
    \UCpaso Se incluye el caso de uso \UCref{CU14} que muestra al usuario una lista de clientes, y
    aqu\'i, el usuario elige un cliente a modificar.
    \UCpaso Se muestra la pantalla \IUref{IUtodo}{Edici\'on de datos de cliente} con los datos del
    cliente previamente seleccionado.
    \UCpaso El usuario edita los campos que requiera a trav\'es de la pantalla mostrada.
    \UCpaso El usuario presiona el bot\'on \IUbutton{Aceptar} para indicar que acepta los cambios
    hechos en los datos del cliente.
    \UCpaso El usuario es redirigido a \IUref{IUtodo}{Pantalla Principal}.
\end{UCtrayectoria}
% \UCpaso \UCactor \IUref \IUbutton \UCref

%Trayectorias Alternativas
%Trayectoria Alternativa
\begin{UCtrayectoriaA}{A}{El usuario presion\'o el bot\'on de Cancelar}
    \UCpaso El usuario es redirigido a \IUref{IUtodo}{Pantalla Principal}.
\end{UCtrayectoriaA}

