\begin{UseCase}{CU14}{Recibir Medicamentos}{
		El proveedor llega con un nuevo lote de medicamentos el cajero debe de tener una opción para poder registrar el ingreso de esos medicamentos
	}
		\UCitem{Versión}{\color{Gray}0.2}
		\UCitem{Autor}{\color{Gray}Aguilera Rosas Landa Enrique}
		\UCitem{Supervisa}{\color{Gray}Zamora Gachuz Jesús Felipe}
		\UCitem{Actor}{Cajero}
		\UCitem{Propósito}{Registro de ingreso de medicamentos en el  inventario de la farmacia.}
		\UCitem{Entradas}{Datos del lote de medicamentos, datos del proveedor}
		\UCitem{Origen}{Teclado}
		\UCitem{Salidas}{Cambio de existencias en el inventrio de los medicamentos}
		\UCitem{Destino}{Pantalla, BD}
		\UCitem{Precondiciones}{El Medicamento debe estar registrado en el sistema, el proveedor debe estar registrado en el sistema.}
		\UCitem{Postcondiciones}{Las unidades del Medicamento deben de aumentar de acuerdo a las unidades ingresadas.}
		\UCitem{Errores}{La pagina sea inaccesible por el momento debido a fallas con los servidores, no tenga registrado el proveedor.}
		\UCitem{Observaciones}{}
		\UCitem{Estado}{Revisión}
	\end{UseCase}
%--------------------------------------
	\begin{UCtrayectoria}{Principal}
		\UCpaso Incluye el caso de uso \UCref{CU0} paso 11
		\UCpaso[\UCactor] Selecciona la opción compras en la \IUref{01}{Pantalla Principal} presionando el botón \IUbutton{Compras}.
		\UCpaso Despliega las opciones de compras en la \IUref{01}
		\UCpaso [\UCactor] Selecciona la opción ingresos en la \IUref{01}{Pantalla Principal} presionando el botón \IUbutton{Ingresos}.
		\UCpaso Despliega en pantalla el formulario \IUref{20}{Formulario Ingreso}.
		\UCpaso [\UCactor] Introduce los datos que pide el formulario \Trayref{A}
		\UCpaso[\UCactor] Guarda el medicamento que recibió presionando el botón \IUbutton{Guardar}
		\UCpaso Redirige al [\UCactor] a la  \IUref{01}{Pantalla Principal}.
	\end{UCtrayectoria}


%-------------------------------------------------------------------------
\begin{UCtrayectoriaA}{A}{Datos vacios.}
			\UCpaso El sistema muestra junto al campo vacio que no puede ser vacio.
			\UCpaso Continúa en el paso 6 del \UCref{CU14}.
		\end{UCtrayectoriaA}
%-------------------------------------------------------------------------
