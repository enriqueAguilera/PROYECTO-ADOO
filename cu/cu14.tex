\begin{UseCase}{CU14}{Recibir Medicamentos}{
		Si el proveedor llega con un nuevo lote de medicamentos el cajero debe de tener una opción para poder ingresar esos medicamentos
	}
		\UCitem{Versión}{\color{Gray}0.1.4}
		\UCitem{Autor}{\color{Gray}Aguilera Rosas Landa Enrique}
		\UCitem{Supervisa}{\color{Gray}Zamora Gachuz Jesús Felipe}
		\UCitem{Actor}{Cajero}
		\UCitem{Propósito}{Control rápido y eficaz sobre el inventario de la farmacia.}
		\UCitem{Entradas}{Lote, ID del Proveedor,  ID del Cajero, Nombre de la Sucursal, Fecha y Hora, ID del Medicamento, Unidades a Recibir del Medicamento, Impuestos Total a Pagar}
		\UCitem{Origen}{Teclado}
		\UCitem{Salidas}{Lote, ID del Proveedor,  ID del Cajero, Nombre de la Sucursal, Fecha y Hora, ID del Medicamento, Unidades a Recibir del Medicamento, Impuestos Total a Pagar.}
		\UCitem{Destino}{Pantalla}
		\UCitem{Precondiciones}{El Medicamento debe estar registrado en el sistema.}
		\UCitem{Postcondiciones}{Las unidades del Medicamento deben de aumentar de acuerdo a las unidades ingresadas .}
		\UCitem{Errores}{La pagina sea inaccesible por el momento debido a fallas con los servidores, Que el empleado tenga su cuenta no este registrado}
		\UCitem{Observaciones}{}
		\UCitem{Estado}{Revisión}
		\UCitem{Viene de}{CU0}
	\end{UseCase}
%--------------------------------------
	\begin{UCtrayectoria}{Principal}
		\UCpaso Incluye el caso de uso \UCref{CU0} 
		\UCpaso[\UCactor] Selecciona La opción Compras en la \IUref{01}{Pantalla Principal} presionando el botón \IUbutton{Compras}.
		\UCpaso Despliega las opciones de Compras en la \IUref{01}
		\UCpaso [\UCactor] Selecciona La opción Ingresos en la \IUref{01}{Pantalla Principal} presionando el botón \IUbutton{Ingresos}.
		\UCpaso Genera y Despliega la \IUref{20}{Formulario Ingreso} con los campos vacíos y listos para llenar.
		\UCpaso[\UCactor] Introduce el Lote del medicamento que recibe, ID del proveedor, ID del cajero, Nombre de la sucursal, Fecha, Id del medicamento, unidades que recibe del medicamento. \Trayref{A}
		\UCpaso[\UCactor] Guarda el medicamento que recibió presionando el botón \IUbutton{Guardar}
		\UCpaso Redirige al [\UCactor] a la  \IUref{01}{Pantalla Principal}.
	\end{UCtrayectoria}


%-------------------------------------------------------------------------
\begin{UCtrayectoriaA}{A}{Empleado no encontrado.}
			\UCpaso Muestra el Mensaje {\bf MSG01-}``Error en la Operación [{\em Proveedor no encontrado}] revisa que los campos sean llenados correctamente en la \IUref{20}{Formulario Ingreso}.''.
			\UCpaso Continúa en el paso 6 del \UCref{CU14}.
		\end{UCtrayectoriaA}
%-------------------------------------------------------------------------
