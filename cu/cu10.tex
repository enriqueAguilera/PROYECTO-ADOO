\begin{UseCase}{CU10}{Agregar Nuevo Medicamento}{
		Se requiere una interfaz para agregar un medicamento de forma sencilla y rápida.
	}
		\UCitem{Versión}{\color{Gray}0.1}
		\UCitem{Autor}{\color{Gray}Vázquez Cruz Fernando Darwin }
		\UCitem{Supervisa}{\color{Gray}}
		\UCitem{Actor}{\hyperlink{Alumno}{Dueño}}
		\UCitem{Propósito}{Que el Dueño pueda agregar un nuevo medicamento a la base de datos para que las sucursales puedan recibir y vender dicho medicamento.}
		\UCitem{Entradas}{Datos del medicamento.}
		\UCitem{Origen}{Teclado, Mouse}
		\UCitem{Salidas}{Mensaje de confirmación de registro de medicamento.}
		\UCitem{Destino}{Pantalla}
		\UCitem{Precondiciones}{El medicamento no debe estar registrado ya en el sistema.}
		\UCitem{Postcondiciones}{Se agregará a la base de datos un nuevo medicamento.}
		\UCitem{Errores}{El medicamento no puede ser agregado.}
		\UCitem{Tipo}{Caso de uso primario.}
		\UCitem{Observaciones}{}
		\UCitem{Estado}{Revisión}
	\end{UseCase}
%--------------------------------------
	\begin{UCtrayectoria}{Principal}
		\UCpaso Se incluye el caso de uso \UCref{CU1}.
		\UCpaso Despliega la \IUref {IU1} {Pantalla Principal}.
		\UCpaso[\UCactor] Ve los medicamentos disponibles dando clic en el\Ibutton {+ Medicamento } .
		\UCpaso Despliega la \IUref {IU17} {Tabla de Medicamentos}.
		\UCpaso[\UCactor] Presiona el \Ibutton {+ Nuevo }.
		\UCpaso Despliega la \IUref {IU5} {Formulario Medicamento}.
		\UCpaso [\UCactor] Llena los campos requeridos para registrar un nuevo medicamento. 
		\UCpaso[\UCactor] Da clic en el \Ibutton {+ Guardar } .
		\UCpaso Despliega el mensaje \bf {+ MSG0}. \Trayref{A}
		\UCpaso Guarda los datos del nuevo empleado y actualiza la lista de empleados.
		\UCpaso[\UCactor] Da clic en el \Ibutton {+ Ok }.
		\UCpaso Re-direcciona al \UCactor a la \IUref {IU17} {Tabla de Medicamentos}.
	
	\end{UCtrayectoria}


		\begin{UCtrayectoriaA}{A}{El Empleado no puede ser agregado}
			\UCpaso Despliega el mensaje \bf {+ MSG1 }.
			\UCpaso[\UCactor] Da clic en el \Ibutton {+ Ok }.
			\UCpaso Continua en el paso 5 del \UCref{CU10}.
		\end{UCtrayectoriaA}

%--------------------------------------