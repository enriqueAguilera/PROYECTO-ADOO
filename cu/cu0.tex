 \begin{UseCase}{CU1}{Control de Acceso}{
		Operación inicial para poder acceder al sistema. De éste Caso de uso se extiende a todos los demás casos de uso.
	}
		\UCitem{Versión}{\color{Gray}0.1}
		\UCitem{Autor}{\color{Gray}Aguilera Rosas Landa Enrique}
		\UCitem{Supervisa}{\color{Gray}.}
		\UCitem{Actor}{Cajero,Supervisor,Dueño}
		\UCitem{Propósito}{Ingresar al Sistema para poder Realizar Transacciones diarias de la farmacia}
		\UCitem{Entradas}{Dirección URL de la pagina web de la farmacia,Contraseña y Correo Electrónico}
		\UCitem{Origen}{Teclado}
		\UCitem{Salidas}{Pantalla principal \IUref{IU1}{Pantalla Principal}}
		\UCitem{Destino}{Pantalla}
		\UCitem{Precondiciones}{Se debe introducir la dirección de la pagina en el navegador de internet }
		\UCitem{Postcondiciones}{El Empleado,Supervisor o Dueño podrá hacer transacciones de su índole}
		\UCitem{Errores}{Que la pagina no este disponible por razones tales de: Error de conexión, Mantenimiento de los servidores, Que no exista el usuario,Que la contraseña este Incorrecta,Que no exista el correo electrónico}
		\UCitem{Observaciones}{}
		\UCitem{Estado}{En revision}
	\end{UseCase}
%--------------------------------------
	\begin{UCtrayectoria}{Principal}
		\UCpaso[\UCactor] Ingresa a la pagina web escribiendo la URL en un navegador.
		\UCpaso Genera Y despliega la Pantalla \IUref{IU0}{Login}
		\UCpaso [\UCactor] Ingresa su Correo Electrónico y Contraseña y presiona el botón Aceptar
		\UCpaso Verifica que el [\UCactor] haya haya proporcionado los datos requeridos en la pantalla \IUref{IU0}{Login}
		\UCpaso Verifica que el correo proporcionado cumpla con el formato ``Ejemplo@ejemplo.com'' \Trayref{A}
		\UCpaso Busca la cuenta asociada al correo ingresado. \Trayref{B}
		\UCpaso Verifica que dicha cuenta este activa. \Trayref{C}
		\UCpaso Verifica que la contraseña ingresada coincida con la contraseña asociada a la cuenta.\Trayref{F}
		\UCpaso Otorga el acceso al sistema
		\UCpaso Muestra la pantalla \IUref{01}{Pantalla Principal}.		
		\UCpaso [\UCactor] Usa el sistema.
		\UCpaso [\UCactor] Solicita cerrar sesión.
		\UCpaso Revoca el acceso.
		\UCpaso Muestra \IUref{IU0}{Login}		
	\end{UCtrayectoria}

%--------------------------------------		
	\begin{UCtrayectoriaA}{A}{El Correo no esta Correcto}
			\UCpaso Muestra el Mensaje en la \IUref{IU0}{Login} {\bf MSG01-}`` Error en la Operación[{\em correo con formato }] Introduzca un correo con el formato Ejemplo@ejemplo.com.''.
			\UCpaso Continúa en el paso 3 del \UCref{CU1}.
		\end{UCtrayectoriaA}
%----------------------------------------
		\begin{UCtrayectoriaA}{B}{El \UCactor no esta registrado}
			\UCpaso Muestra el Mensaje en la \IUref{IU0}{Login} {\bf MSG01-}``Error en la operación [{\em Usuario no Encontrado}] El usuario y/o contraseña no existen  .''.
			\UCpaso[\UCactor] Oprime el botón \IUbutton{Aceptar}.
			\UCpaso[] Continua en el paso 3 del \UCref{CU1}.
		\end{UCtrayectoriaA}		
%--------------------------------------
		\begin{UCtrayectoriaA}{C}{La cuenta a la que intenta acceder no esta activa}
			\UCpaso Muestra el Mensaje en la \IUref{IU0}{Login} {\bf MSG01-}``Error en la operación [{\em Cuenta Desactivada}] Contacta con el Dueño para resolver el problema .''.
			\UCpaso[\UCactor] Oprime el botón \IUbutton{Aceptar}
			\UCpaso Continua en el paso 3 del \UCref{CU1}.
		\end{UCtrayectoriaA}
%--------------------------------------
		\begin{UCtrayectoriaA}{F}{La Contraseña es incorrecta}
			\UCpaso Muestra el Mensaje en la \IUref{IU0}{Login} {\bf MSG01-}``Error en la Operación [{\em Contraseña invalida}] La contraseña ingresada no coincide con la cuenta.''.
			\UCpaso Continúa en el paso 2 del \UCref{CU1}.
		\end{UCtrayectoriaA}
