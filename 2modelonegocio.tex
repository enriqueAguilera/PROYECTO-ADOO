\chapter{Modelo del Negocio}
\section{Contexto}
La Franquicia de Farmacias "Fran Farmacias" se dedica a la venta de medicamentos que requieren o no receta. La franquicia consta con un único dueño y múltiples empleados atendiendo las sucursales, de la misma manera el supervisor de la sucursal en la que esta asignado, verifica en la apertura y al cierre del turno que la operación de la sucursal sea de la mejor manera posible y lo mismo pasa en las diferentes sucursales de la franquicia\\

\section{Términos del Negocio}

Cliente: Se refiere a todas las personas físicas y morales que Compran medicamentos ya sea que estos sean clientes registrados en el sistema o sin registrar.\\

Cliente Preferente: Se refiere a todas aquellas personas físicas y morales que Compran medicamentos y que están registrados en el sistema, estos clientes cuentan con un monedero electrónico


Dosis: Cantidad a ingerir o suministrar expresado en unidades de volumen o peso(gramos) por unidad de un Medicamento.\\

Dueño: Es propietario de la farmacia y de todas sus sucursales, se encarga de la contratación de todos los Empleados, es el Empleado con mayor rango en “Fran Farmacias”.\\

Empleado: Se refiere a cualquier persona que labore en la empresa exceptuando al Dueño.\\

Estado: Hace referencia a los elementos, definiendo un elemento como: Cliente, Sucursal, Medicamento, Empleado, Paquete de descuento y Proveedor, Donde dicho elemento puede ser manipulado en operaciones de lectura y escritura (Estado activado) o solo lectura(Estado desactivado)\\

Ingrediente Activo: Sustancia del Medicamento con composición química exactamente conocida y que es capaz de producir efectos o cambios sobre el cuerpo de quien lo consume.\\

Laboratorio Farmacéutico: Aquellas personas físicas o jurídicas que, previamente autorizadas por la Administración competente, fabriquen de forma industrial Medicamentos o participen en alguna de sus fases.\\

Lote: Es una clave de identificación de los Medicamentos de un mismo proceso de fabricación.Tiene valor para el Laboratorio Farmacéutico. Consta de una letra , que indica el año de fabricación, y de un número.\\

Medicamento: Producto que sirve para curar, prevenir una enfermedad, para reducir sus efectos sobre el organismo o para aliviar un dolor físico.\\

Monedero electrónico: Dinero abonado por una devolución a la cuenta del cliente preferente.\\

Paquete De Descuento: Son paquetes que añade el dueño cuando se esta por caducar un medicamento, o es temporada en la que un medicamento tiene un rango de venta mayor que otros y se aprovecha para hacer un descuento de esos medicamentos, al conjunto de medicamentos que se van a poner en descuento es un: paquete de Descuento.\\

Presentación Del Medicamento: Es el tipo de Presentación del medicamento, los cuales son:\\
1.-Presentación Solida: Polvos, Cápsulas, Tabletas o Comprimidos, Píldoras, Grageas, Supositorios, Óvulos\\
2.-Presentaciones Semisolidas: Pomadas o Cremas\\
3.-Presentaciones Liquidas: Soluciones, Jarabes, Colirios, Lociones, Linimentos, Elixir, Enemas, Inhalaciones, Aerosoles\\

Sucursal: Una extensión de la farmacia donde se opera como la farmacia original.
Esta Sucursal tiene nombre Único \\

Supervisor: (es un tipo de Empleado) Es un empleado con mayor Jerarquía que Empleado pero menor que Dueño.\\

Venta: Son los datos que se guardan de una venta de medicamento realizada.\\

Vía de Administración:  Es la forma en la que el cliente se tiene que aplicar el medicamento.entre estas están: Vía Digestiva, Vía Oral, Vía Sublingual, Gastroenteritis, Vía Rectal, Vía Parental, Vía Respiratoria, Vía Tópica, Vía Oftálmica, Vía Ótica, Vía Transdérmica.\\
\newpage
\section{Modelo de Dominio del problema}

	El modelo del dominio del problema se muestra en la figura~\ref{fig:modeloDeDominio}, a continuación se describen cada una de las entidades y sus relaciones.
	
\begin{figure}[htbp!]
	\begin{center}
		\includegraphics[width=.9\textwidth]{images/diagramaRelacional}
		\caption{Modelo del dominio del problema}
		\label{fig:modeloDeDominio}
	\end{center}
\end{figure}
%---------------------detalle de las entidades

%- - - - - - - - - - - - - - - - - - - - - - - - - - - - - 
\newenvironment{cdtEntidad}[2]{%
	\def \varBusinessEntityId{#2}%
	\hypertarget{#1}{\hspace{1pt}}%
	\newline%
	\noindent{\includegraphics[width=\textwidth]{images/uc/classRule}}%
	\vspace{-25pt}%
	\subsection{Entidad: #2}%
	\noindent\begin{longtable}{|p{.2\textwidth}| p{.15\textwidth} | p{.46\textwidth} |p{.08\textwidth} |}%
	\hline%
	\multicolumn{4}{|c|}{{\cellcolor{colorSecundario}\color{white}Atributos}}\\ \hline%
	{\cellcolor{colorAgua}Nombre} &%
	{\cellcolor{colorAgua}Tipo} &%
	{\cellcolor{colorAgua}Descripción} &%
	{\cellcolor{colorAgua}Requerido}%
	\\ \hline%
	\endhead%
}{%
	\end{longtable}%
}

\newcommand{\brAttr}[5]{%
	{\bf\hypertarget{\varBusinessEntityId:#1}{#2}} & {\em{#3}} & {#4} & #5 \\\hline
}

\newcommand{\cdtEntityRelSection}{%
	\multicolumn{4}{|c|}{{\cellcolor{colorSecundario}\color{white}Relaciones}}\\ \hline%
	{\cellcolor{colorAgua}Tipo de relación} &%
	{\cellcolor{colorAgua}Entidad} &%
	\multicolumn{2}{|c|}{{\cellcolor{colorAgua}Rol}}
	\\ \hline%
}

\newcommand{\brRelComposition}{{\color{colorPrincipal}$\Diamondblack$\hspace{-1pt}---Composición}}
\newcommand{\brRelAgregation}{{\color{colorPrincipal}$\Diamond$\hspace{-1pt}---Agregación}}
\newcommand{\brRelGeneralization}{{\color{colorPrincipal}$\lhd$\hspace{-1pt}---Generalización}}

\newcommand{\brRel}[3]{%
	{\em{#1}} & {\bf{#2}} & \multicolumn{2}{|l|}{#3}\\\hline
}
\newpage


%- - - - - -------------------------------------------------
\begin{cdtEntidad}{Dueño}{Dueño}
\brAttr{ID}{ID}{int}{Número de registro utilizado para identificar al Dueño}{Sí}
	\brAttr{nombre}{Nombre}{String}
		{Nombre o nombres del Dueño.}{Sí}
		
	\brAttr{ApellidoPaterno}{Apellido Paterno}{String}
		{Apellido Paterno del Dueño.}{Sí}
		
	\brAttr{ApellidoMaterno}{ApellidoMaterno}{String}
		{Apellido Materno del Dueño.}{No}
		
	\brAttr{Email}{Email}{String}
		{Correo del Dueño para enviar información Laboral y entrar al sistema junto con su Clave.}{Sí}
		
	\brAttr{Clave}{Clave}{String}
		{Forma de permitir al Dueño ingresar en el sistema junto con su correo.}{Sí}
		
\end{cdtEntidad}

%- - - - - -------------------------------------------------
\begin{cdtEntidad}{Supervisor}{Supervisor}
\brAttr{registro}{Registro}{int}{Número de registro utilizado para identificar al supervisor}{Sí}
	\brAttr{nombre}{Nombre}{String}
		{Nombre o nombres del Supervisor.}{Sí}
		
	\brAttr{primerApellido}{Primer apellido}{String}
		{Apellido Paterno del Supervisor.}{Sí}
		
	\brAttr{segundoApellido}{Segundo apellido}{String}
		{Apellido Materno del Supervisor.}{No}
		
	\brAttr{Email}{Email}{String}
		{Correo del Empleado para enviar información Laboral y identificar al Supervisor en el sistema con sus respectivos permisos.}{Sí}

	\brAttr{telefono}{Teléfono}{int}
		{Teléfono para contactar al Supervisor.}{Sí}
		
	\brAttr{Contraseña}{Contraseña}{String}
		{Forma de permitir al Supervisor ingresar en el sistema junto con su correo.}{Sí}
		
	\brAttr{Estado}{Estado}{TINYINT}
		{Estado del supervisor, puede estar activado o desactivado (un tipo de dato TINYINT representa un carácter , por razones de eficiencia se utiliza este tipo de dato en vez de un booleano).}{Sí}	
	
	\brAttr{Sucursal}{Sucursal}{Sucursal}
		{Sucursal de la que esta a cargo el supervisor.}{Sí}

		\cdtEntityRelSection
	\brRel{\brRelAgregation}{Sucursal}{El Supervisor Es un tipo especializado de Empleado}
\end{cdtEntidad}



%- - - - - -------------------------------------------------
\begin{cdtEntidad}{Cajero}{Cajero}
\brAttr{ID}{ID}{int}{Número de registro utilizado para identificar al Cajero}{Sí}
	\brAttr{nombre}{Nombre}{String}
		{Nombre o nombres del Cajero.}{Sí}
		
	\brAttr{ApellidoPaterno}{Apellido Paterno}{String}
		{Apellido Paterno del Cajero.}{Sí}
		
	\brAttr{ApellidoMaterno}{ApellidoMaterno}{String}
		{Apellido Materno del Cajero.}{No}
		
	\brAttr{Salaro}{Salariol}{Float}
		{Salario a que cobra el cajero.}{Sí}
		
	\brAttr{telefono}{Teléfono}{String}
		{Teléfono para contactar al Cajero.}{Sí}
		
	\brAttr{Email}{Email}{String}
		{Correo del Cajero para enviar información Laboral y entrar al sistema junto con su Clave.}{Sí}
		
	\brAttr{Clave}{Clave}{String}
		{Forma de permitir al Cajero ingresar en el sistema junto con su correo.}{Sí}
	
	\brAttr{Estado}{Estado}{TINYINT}
		{Estado del cajero, puede estar activado o desactivado (un tipo de dato TINYINT representa un carácter , por razones de eficiencia se utiliza este tipo de dato en vez de un booleano).}{Sí}	
	
	\brAttr{Sucursal}{Sucursal}{Sucursal}
		{Sucursal de la que esta a cargo el supervisor.}{Sí}

		\cdtEntityRelSection
	\brRel{\brRelAgregation}{Sucursal}{El cajero Es un tipo de Empleado}
\end{cdtEntidad}

%- - - - - - - - - - - - - - - - - - - - - - - - - - - - - 
\begin{cdtEntidad}{Cliente}{Cliente}
	\brAttr{ID}{ID}{int}{Número de registro utilizado para identificar al Cliente}{Sí}
	\brAttr{nombre}{Nombre}{String}
		{Nombre o nombres del Cliente.}{Sí}
		
	\brAttr{ApellidoPaterno}{Apellido Paterno}{String}
		{Apellido Paterno del Cliente.}{Sí}
		
	\brAttr{ApellidoMaterno}{ApellidoMaterno}{String}
		{Apellido Materno del Cliente.}{No}
		
	\brAttr{Email}{Email}{String}
		{Correo del cliente para enviar información sobre promociones.}{Sí}
		
	\brAttr{telefono}{Teléfono}{String}
		{Teléfono para contactar al Cliente.}{Sí}
		
	\brAttr{Tarjeta}{Tarjeta}{int}
		{forma única de hacer devoluciones sobre compras hechas a los clientes.}{si}

	\brAttr{Estado}{Estado}{TINYINT}
		{Estado del Cliente, puede estar activado o desactivado (un tipo de dato TINYINT representa un carácter , por razones de eficiencia se utiliza este tipo de dato en vez de un booleano).}{Sí}	
		
		\cdtEntityRelSection
	\brRel{\brRelAgregation}{Cliente}{El Cliente Compra en Sucursal}
\end{cdtEntidad}
%---------------------------------------------------------

\begin{cdtEntidad}{Medicamento}{Medicamento}
	\brAttr{ID}{ID}{Int}{Identificador del Medicamento.}{Sí}
	
	\brAttr{IDViaAdministracion}{IDViaAdministracion}{Int}{Identificador de la vía de Administración del Medicamento.}{Sí}
	
	\brAttr{IDPaqueteDescuento}{IDPaqueteDescuento}{Int}{Identificador del paquete de descuento al que pertenece el Medicamento.}{Sí}

	\brAttr{IDPresentacion}{IDPresentaicon}{Int}{Identificador del tipo de presentación del Medicamento.}{Sí}
	
	\brAttr{Nombre}{Nombre}{String}{Nombre del Medicamento.}{Sí}
	
	\brAttr{Marca}{Marca}{String}{Marca del medicamento}{Sí}
	
	\brAttr{FechaCaducidad}{FechaCaducidad}{DATE}{La fecha de caducidad del medicamento}{si}
	
	\brAttr{Lote}{Lote}{Int}{el numero de lote del medicamento}{Sí}
	
	\brAttr{ViaDeAdministracion}{ViaDeAdministracion}{String}
		{Indica la forma en la que el medicamento debe ser ingerido}{si}
		
	\brAttr{Advertencias}{Advertencias}{string}
		{Las precauciones que se deben tomar en cuenta con el medicamento}{si}
		
	\brAttr{PrecioPublico}{PrecioPublico}{int}
		{El precio al que se le vende al cliente.}{si}
		
	\brAttr{PrecioCompra}{PrecioCompra}{Int}
		{El precio al que se compra el medicamento al proveedor}{si}
		
	\brAttr{noIngedientesActivos}{noIngredientesActivos}{int}
		{número de ingredientes activos que contiene el medicamento}{si}

		\cdtEntityRelSection
	\brRel{\brRelComposition}{Medicamento}{El Medicamento Esta compuesto por IngredienteActivo}
\end{cdtEntidad}
%- - - - - - - - - - - - - - - - - - - - - - - - - - - - - 
\begin{cdtEntidad}{IngredienteActivo}{IngredienteActivo}
	\brAttr{ID}{ID}{Int}
		{El identificador al que pertenece el ingrediente activo}{Sí}
	\brAttr{Nombre}{Nombre}{String}
		{El Nombre del ingrediente activo}{Sí}
	\brAttr{Cantidad}{Cantidad}{String}
		{La cantidad del ingrediente activo presente en el Medicamento}{Sí}

\end{cdtEntidad}
%---------------------------------------------------------

%- - - - - - - - - - - - - - - - - - - - - - - - - - - - - 
\begin{cdtEntidad}{Presentación}{Presentación}
	\brAttr{ID}{ID}{Int}
		{El identificador al que pertenece la Presentación Del Medicamento}{Sí}
	\brAttr{Nombre}{Nombre}{String}
		{El Nombre del tipo de presentación}{Sí}

\end{cdtEntidad}

%- - - - - - - - - - - - - - - - - - - - - - - - - - - - - 
\begin{cdtEntidad}{ViaAdministracion}{ViaAdministracion}
	\brAttr{ID}{ID}{Int}
		{El identificador al que pertenece a la Vía de Administración}{Sí}
	\brAttr{Nombre}{Nombre}{String}
		{El Nombre del tipo de Vía de Administración}{Sí}

\end{cdtEntidad}

%- - - - - - - - - - - - - - - - - - - - - - - - - - - - - 
\begin{cdtEntidad}{Lote}{Lote}
	\brAttr{ID}{ID}{Int}
		{El identificador al que pertenece el Lote}{Sí}
	\brAttr{Nombre}{Nombre}{String}
		{El Nombre compuesto del lote}{Sí}
	\brAttr{FechaCaducidad}{FechaCaducidad}{DATE}
		{Fecha en la que el medicamento caduca}{Sí}
	\brAttr{PrecioCompra}{PrecioCompra}{DATE}
		{Precio en el que se compro el Lote de medicamento}{Sí}
	\brAttr{Cantidad}{Cantidad}{DATE}
		{Cantidad de medicamento que se registro por lote}{Sí}
\end{cdtEntidad}

%------------------------------------------------------------
\begin{cdtEntidad}{PaqueteDescuento}{PaqueteDescuento}
	\brAttr{ID}{ID}Int
		{Identificador del descuento.}{Sí}
	\brAttr{porcentajeDescuento}{porcentajeDescuento}{Int}
		{Descuento de porcentaje(\%) a aplicar sobre los medicamentos}{si}
	\brAttr{FechaInicio}{FechaInicio}{DATE}
		{La fecha en la que se empieza a aplicar el descuento}{si}
	\brAttr{FechaFin}{FechaFin}{DATE}
		{Fecha limite para aplicar Descuento}{Sí}
\end{cdtEntidad}

%- - - - - - - - - - - - - - - - - - - - - - - - - - - - - 
\begin{cdtEntidad}{Venta}{Venta}

	\brAttr{idVenta}{idVenta}{int}
		{Identificador del la venta}{si}
		
	\brAttr{IDCliente}{IDCliente}{Int}
		{Cliente al que se le vendió si es que este esta registrado en el sistema.}{si}
		
	\brAttr{IDSucursal}{IDSucursal}{Int}
		{Sucursal en la que se realizó la venta}{si}
		
	\brAttr{IDEmpleado}{IDEmpleado}{Int}
		{Empleado que realizó la venta}{si}
		
	\brAttr{Condición}{Condición}{TINYINT}
		{Estado de la venta, puede estar activa o cancelada (un tipo de dato TINYINT representa un carácter , por razones de eficiencia se utiliza este tipo de dato en vez de un booleano).}{Sí}	

	\brAttr{Comprobante}{Comprobante}{String}
		{Comprobante de la venta}{si}
		
	\brAttr{noComprobante}{noComprobante}{String}
		{Identificador de Comprobante de la venta}{si}
		
	\brAttr{Impuesto}{Impuesto}{Float}
		{Impuesto de la venta}{si}
		
	\brAttr{TotalVenta}{TotalVenta}{Float}
		{Costo total de la venta realizada}{si}
	
\end{cdtEntidad}
%- - - - - - - - - - - - - - - - - - - - - - - - - - - - - 

%- - - - - - - - - - - - - - - - - - - - - - - - - - - - - 
\begin{cdtEntidad}{DetalleVenta}{DetalleVenta}

	\brAttr{idDetalleVen}{idDetalleVen}{int}
		{Identificador del Detalle de la venta}{si}
		
	\brAttr{IDVenta}{IDVenta}{Int}
		{Identificador de la Venta}{si}
		
	\brAttr{PrecioVenta}{PrecioVenta}{Int}
		{Precio de la venta realizada}{si}
		
	\brAttr{Cantidad}{Cantidad}{Int}
		{Cantidad por la cual se realizó la venta}{si}
		
	\brAttr{Descuento}{Descuento}{Float}
		{Descuento aplicado sobre la venta.}{Sí}	
	
\end{cdtEntidad}
%- - - - - - - - - - - - - - - - - - - - - - - - - - - - - 

%- - - - - -------------------------------------------------
\begin{cdtEntidad}{Proveedor}{Proveedor}
\brAttr{ID}{ID}{int}{Número de registro utilizado para identificar al proveedor}{Sí}
	\brAttr{nombre}{Nombre}{String}
		{Nombre o nombres del proveedor.}{Sí}
		
	\brAttr{ApellidoPaterno}{Apellido Paterno}{String}
		{Apellido Paterno del proveedor.}{Sí}
		
	\brAttr{ApellidoMaterno}{ApellidoMaterno}{String}
		{Apellido Materno del proveedor.}{No}

	\brAttr{Email}{Email}{String}
		{Correo del proveedor para poder contactar con el.}{Sí}
		
	\brAttr{telefono}{Teléfono}{String}
		{Teléfono para contactar al proveedor.}{Sí}
		
	\brAttr{RFC}{RFC}{String}
		{RFC del proveedor.}{Sí}
	
	\brAttr{Estado}{Estado}{TINYINT}
		{Estado del proveedor, puede estar activado o desactivado (un tipo de dato TINYINT representa un carácter , por razones de eficiencia se utiliza este tipo de dato en vez de un booleano).}{Sí}	
\end{cdtEntidad}

%- - - - - - - - - - - - - - - - - - - - - - - - - - - - - 

\begin{cdtEntidad}{Ingreso}{Ingreso}
\brAttr{ID}{ID}{int}{Número de registro utilizado para identificar el ingreso}{Sí}
	\brAttr{FechaHora}{FechaHora}{DATETIME}
		{Fecha y Hora del ingreso.}{Sí}
		
	\brAttr{TotalCompra}{TotalCompra}{Int}
		{Cantidad total de la compra a ingresar.}{Sí}
		
	\brAttr{Condición}{Condición}{TINYINT}
		{Condición de la compra a ingresar.}{No}

	\brAttr{Impuesto}{Impuesto}{Double}
		{Impuesto Aplicado sobre la compra a ingresar.}{Sí}
		
	\brAttr{Comprobante}{Comprobante}{String}
		{Comprobante de la compra realizada.}{Sí}
		
	\brAttr{RFC}{RFC}{String}
		{RFC del proveedor.}{Sí}
	
	\brAttr{Estado}{Estado}{TINYINT}
		{Estado del proveedor, puede estar activado o desactivado (un tipo de dato TINYINT representa un carácter , por razones de eficiencia se utiliza este tipo de dato en vez de un booleano).}{Sí}	
\end{cdtEntidad}








%- - - - - - - - - - - - - - - - - - - - - - - - - - - - - 
\begin{cdtEntidad}{Sucursal}{Sucursal}

	\brAttr{idSucursal}{idSucursal}{Int}
		{Identificador único de la sucursal}{Sí}
		
	\brAttr{Nombre}{Nombre}{String}
		{Nombre de la sucursal.}{Sí}

	\brAttr{Dirección}{Dirección}{String}
		{Localización de la sucursal.}{Sí}
	
	\brAttr{Teléfono}{Teléfono}{String}
		{Teléfono de contacto de la sucursal.}{Sí}
		
	\brAttr{EstadoRepublica}{EstadoRepublica}{String}
		{Estado de la República donde esta ubicada la sucursal.}{Sí}
		
	\brAttr{noComprobante}{noComprobante}{String}
		{Identificador del Comprobante sobre la compra realizada.}{Sí}	

\end{cdtEntidad}
%- - - - - - - - - - - - - - - - - - - - - - - - - - - - - 

\begin{cdtEntidad}{Horarios}{Horarios}

	\brAttr{ID}{ID}{Int}
		{Identificador único del Horario}{Sí}
		
	\brAttr{Nombre}{Nombre}{String}
		{Nombre del Horario.}{Sí}

	\brAttr{HoraInicio}{HoraInicio}{DATETIME}
		{Hora de Inicio del Turno.}{Sí}
	
	\brAttr{HoraFinal}{HoraFinal}{DATETIME}
		{Hora de Fin del Turno.}{Sí}
		
\end{cdtEntidad}
%- - - - - - - - - - - - - - - - - - - - - - - - - - - - - 




%------------------------------------------------------
\newpage
\section{Reglas del Negocio}
\begin{BussinesRule}{BR1}{Control de Acceso.}
	\BRitem[Tipo:] Regla de Autorizacion. 
				
	\BRitem[Clase:] Habilitadora. 
	\BRitem[Nivel:] Control. % Otras opciones para nivel: Control, Influencia.
	\BRitem[Descripción:]	El acceso al Sistema debe ser manejado mediante un control de acceso, el cual solo permitira ingresar al sistema una vez que el Empleado ingrese de manera correcta su Usuario y Contraseña.
	\BRitem[Motivación:] Evitar posibles fraudes en las operaciones y funcionamiento del sistema de la farmacia.
	\BRitem[Ejemplo positivo:]
	\BRitem[Ejemplo negativo:] 
	\BRitem[Referenciado por:] \hyperlink{CU0}{CU0}.
\end{BussinesRule}

\begin{BussinesRule}{BR2}{Eliminar Un Medicamento.}
	\BRitem[Tipo:] Regla de Autorizacion. 
	\BRitem[Clase:] Habilitadora. 
	\BRitem[Nivel:] Control. % Otras opciones para nivel: Control, Influencia.
	\BRitem[Descripción:]	Solo el dueño puede eliminar medicamentos que han sido registrados en el sistema.
	\BRitem[Motivación:] Evitar posibles fraudes en las operaciones y funcionamiento del sistema de la farmacia.
	\BRitem[Ejemplo positivo:]
	\BRitem[Ejemplo negativo:]
	\BRitem[Referenciado por:] \hyperlink{CU3}{CU3}.
\end{BussinesRule}
%------------------------------------------------------------------------------------
\begin{BussinesRule}{BR3}{Realizar Una Venta.}
	\BRitem[Tipo:] Regla de Operacion. 
	\BRitem[Clase:] Cronometrada. 
	\BRitem[Nivel:] Control. % Otras opciones para nivel: Control, Influencia.
	\BRitem[Descripción:] El numero de medicamentos en el almacen debe de ser mayor a la cantidad que se realiza al momento de la venta.
	\BRitem[Motivación:] Evitar problemas con el inventario de la Farmacia .
\BRitem[Sentencia:] 
		\begin{displaymath}
			Inventario>=Venta\\
			Inventario= CantidadInventario-Venta;
		\end{displaymath}
	\BRitem[Ejemplo positivo:] 
	\BRitem[Ejemplo negativo:] 
	\BRitem[Referenciado por:] \hyperlink{CU5}{CU5}.
\end{BussinesRule}

%------------------------------------------------------------------------------------
\begin{BussinesRule}{BR4}{Aplicar Un Descuento.}
	\BRitem[Tipo:] Regla de Operacion. 
	\BRitem[Clase:] Ejecutive. 
	\BRitem[Nivel:] Control. % Otras opciones para nivel: Control, Influencia.
	\BRitem[Descripción:] Bajo ciertas Condiciones se aplicaran descuentos sobre ciertos Medicamentos.
	\BRitem[Motivación:] Evitar perdida de dinero al dejar que los medicamentos que casi no tienen salida de almacen se queden guardados y terminen caducando.
\BRitem[Sentencia:] 
		\begin{displaymath}
			MedicamentoConDescuento=Medicamento*Descuento\\
		\end{displaymath}
	\BRitem[Ejemplo positivo:] 
	\BRitem[Ejemplo negativo:] 
	\BRitem[Referenciado por:] \hyperlink{CU8}{CU8}.
\end{BussinesRule}

%------------------------------------------------------------------------------------
\begin{BussinesRule}{BR5}{Abrir Turno de Caja.}
	\BRitem[Tipo:] Regla de Integridad Referencial. 
	\BRitem[Clase:] Ejecutive. 
	\BRitem[Nivel:] Control. % Otras opciones para nivel: Control, Influencia.
	\BRitem[Descripción:]Al inicio de cada jornada laboral el supervisor debe ser el que Abre el turno de caja para iniciar con las operaciones diarias de la farmacia.
	\BRitem[Motivación:] Tener un Reporte de ventas que sea relevante y descriptivo sobre las operaciones de cierto periodo en la farmacia.
	\BRitem[Ejemplo positivo:]
	\BRitem[Ejemplo negativo:] 
	\BRitem[Referenciado por:] \hyperlink{CU23}{CU23}.
\end{BussinesRule}

%------------------------------------------------------------------------------------
\begin{BussinesRule}{BR6}{Cerrar Turno de Caja.}
	\BRitem[Tipo:] Regla de Integridad Referencial. 
	\BRitem[Clase:] Ejecutive. 
	\BRitem[Nivel:] Control. % Otras opciones para nivel: Control, Influencia.
	\BRitem[Descripción:]Al final de cada jornada laboral el supervisor se encargara del cierre de caja.
	\BRitem[Motivación:] Evitar Perdias de dinero en la farmacia y que al final de cada turno de caja las ventas realizadas coincidan con el reporte de ventas.
	\BRitem[Ejemplo positivo:]
	\BRitem[Ejemplo negativo:] 
	\BRitem[Referenciado por:] \hyperlink{CU24}{CU24}.
\end{BussinesRule}
\newpage


\section{Procesos del Negocio}













