\chapter{Introducción}
Este documento contiene la Especificación del proyecto “Fran Farmacias” correspondiente al trabajo realizado en el
semestre 2017-2018-1 para la materia de la Escuela Superior De Computo. Análisis y diseño orientado a objetos en el
grupo 2CV9 por el equipo Lo Que Sea.com

\section{Presentación}

Este documento contiene la especificación de los requerimientos del usuario y del sistema a desarrollar.
Tiene como objetivo establecer la naturaleza y funciones del sistema para su evaluacion al final del semestre. 
Este documento debe ser aprobado por los principales responsables del proyecto.
Este documento es el Documento de Análisis del proyecto “Fran Farmacias”.

\section{para qué sirve.}
\section{Que contiene.}
\section{Organización}
\section{Aplicabilidad}

\section{Notación, Simbolos y Convenciones Utilizadas}
Los requerimientos funcionales utilizan una clave RFX, donde:\\
X Es un numero consecutivo: 1, 2, 3, ... \\
RF Es la clave para todos los Requerimientos Funcionales.\\\\
Ademas, para los requerimientos funcionales se usan las siguientes abreviaciones.\\
\begin{center}
Id Identificador del requerimiento.\\
Pri. Prioridad\\
Ref. Referencia a los Requerimientos de usuario.\\
MA Prioridad Muy Alta.\\
A Prioridad Alta.\\
M Prioridad Media.\\
B Prioridad Baja.\\
MB Prioridad Muy Baja.\\
\end{center}
Los requerimientos del usuario utilizan una clave RUX, donde:\\
X Es un numero consecutivo: 1, 2, 3, ...\\
RU Es la clave para todos los Requerimientos del Usuario.\\\\
Para los requerimientos del usuario se usan las siguientes abreviaciones.\\
\begin{center}
Los Casos de Uso utilizan una clave CUX, donde:\\
X Es un numero consecutivo: 1, 2, 3, ... \\
CU Es la clave para todos los Casos de Uso.\\
\end{center}
